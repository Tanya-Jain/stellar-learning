\chapter{Introduction}
\begin{epigraphs}
\qitem{\textquotedblleft Thousands of candles can be lit from a single candle, and the life of the candle will not be shortened. Happiness never decreases by being shared.\textquotedblright}%
      {Gautam Buddha, (Founder of Buddhism, 563 - 483 BC)}
\qitem{\textquotedblleft I alone cannot change the world, but I can cast a stone across the waters to create many ripples.\textquotedblright}%
      {Mother Teresa}
\end{epigraphs}      

ASET ALiAS is a technical club cum community of Amity School of Engineering and Technology's (ASET) Computer Science and Engineering Department. Amity Linux Assistance Sapience (ALiAS) is a community of developers and designers who collaborate for shared learning and hacking technologies. Since starting back in 2010, our main aim is to foster the usage of GNU/Linux without which a computer science student remains incomplete. ALiAS is now a platform for upcoming developers and designers for finding exposure by meeting the people working in related industries, learning various languages and becoming a better developer or designer. \par

\section{The Need for a mobile application}

With over 4000+ students in the Computer Science and Engineering department at Amity University, Noida solely, as well as a new ALiAS student chapter opened at Amity University, Lucknow and students from local meetup groups, we see the need for a common place where management of such a large community and proper dissemination of information becomes thereby possible. \par

The club organizes various events that has nurtured the formation of a community of students within the campus as well as outside, some of which include: 
\begin{enumerate}
    \item On-site weekly events and meet-ups on Thursday
    \item Webinars by people in industry to teach about the filed on ALiAS' Youtube channel
    \item Weekly VOIP Conference calls over Skype (mostly) with alumni and student mentors to discuss about projects and self paced learning progress, to have a tailored learning and teaching experience
    \item ALiAS meets at local technical community meet-ups and conferences like PyDelhi Conference, PyCon India
\end{enumerate}
\par \bigskip
Organizing multiple events brings certain responsibilities that must be looked after on time and efficiently:
\begin{enumerate}
    \item Dissemination of events' information must be on time and accurate. In case there is some change in any way, community must be updated immediately about it.
    \item There are a lot of free and open documents available for read that are recommended for self-paced learning, getting started as well as those answering general frequently asked questions. Availability of such documents curated in one place is beneficial for both the giver and the taker. This would a also save time for a lot of people. 
    \item Even on being informed multiple times about person of contact for a required person, people either hesitate or do not remember. Having them well written with social media links for communication would help potential mentees to solve their problems and move forward in their quest to learn.
\end{enumerate}

\section{Which technology to choose?}
While university's course focused on Android app development on Java, I had been on a look out for various methods that would be efficient, well documented and would aid in developing code that can also generate good design. After considering app development with Java, python and JavaScript, I chose the latter which is more reasoned out in chapter \ref{chap:devTools}. For cross-platform, I preferred native over hybrid as it is more efficient, has quicker response time and does not look out of place as other available ones.

\section{Document Structure}
The introductory chapter has laid out a more in-depth base on the objective of the project done. It introduces the readers to cross-platform native app development which is not favourable for both programmers and designers. The tools used in the development of the mobile application and its application will be discussed in the upcoming chapter.
Chapter \ref{chap:devTools} lays down the various tools that have aided in the successful development of the React Native mobile application. It covers the materials including the programming languages and their framework, text-editors, development environment used, as well as the methods in the utilization of these materials.
The well laid formatting and structure of this document is a result of the utilities provided by the tool, \LaTeX. 