\begin{abstract}
The aim of the project was to find a method to develop Bayesian A/B Tester and understand the mathematical and scientific reasons behind the working of the A/B analyzer. The Beta-Bernoulli model is a binary outcome model connected in the A/B testing setting, where the objective of derivation is understanding the likelihood that the test assembly performs superior to the control assembly. The Bayesian A/B analyzer utilizes a Beta distribution as the prior for the success probability.

Bayesian inference comprises of initially examining the prior belief and the likelihood of the effects that shall take place, then updating the earlier with recent information for a refreshed prior. For instance, for a determined conversion rate of 7\%, it may seem practically likely for the change to take place for the benefit after the test by 7\%. T if our transformation rate is 5\%, we may state that it's sensibly likely that a here is also, a 93\% chance for there to be no impact after the test, which is of higher likelihood. Moreover, it indicates that it is not possible for the conversion rates to shoot up more than 25\%.

As the information begins coming in, the set beliefs are refreshed. In the event that the updated information indicates towards an increased conversion rate, the gauge of the impact shifts upwards from the prior. Constant gathering of information helps in validation of a result and moving further away from the prior. Posterior probability distribution is achieved as the end product of the treatment performed on the prior data.

Finally, the A/B tester is tested using Statistical significance based on Frequentist inference, an approach less preferred over Bayesian approach for A/B Testing due to lack in hypothesis generation.
\end{abstract}