\chapter{Introduction}
\begin{epigraphs}
\qitem{Thousands of candles can be lit from a single candle, and the life of the candle will not be shortened. Happiness never decreases by being shared.}%
      {Gautam Buddha, (Founder of Buddhism, 563 - 483 BC)}
\qitem{Whenever I found out anything remarkable, I have thought it my duty to put down my discovery on paper, so that all ingenious people might be informed thereof.}%
      {Antonie van Leeuwenhoek (Dutch Biologist, 1632-1723)}
\end{epigraphs}
Sharing is a way of life and in a broader sense, sharing is what makes us human. Sharing of knowledge and sense of collaboration have propelled the development of human race and made it the mighitiest one, the Earth has seen so far. From the ancient times of \emph{rishis} to the scientists at CERN in the present times, sharing of knowledge has been crucial to advancements. In the present day, the open source movement has become synonymous to the ideology of free sharing. \par
\section{Computers, Internet, Sharing and Collaboration}
Ever since humans created computers and connected them into a network, sharing has not only been faster but easier too. Today not just the elite scientific community but the general crowd shares data eg Photos, Videos, Music text over the ever growing Internet. There are loads of applications which help the user to share stuff online. But all of it seems to be futile when it comes to files.  Lets dive into the realm and see how things are presently. \par

\subsection{EMail}
When EMail or electronic mail came into existence almost three decades ago, it was barely used anywhere except few universities and big industries. Whereas, the number of EMail users worldwide was estimated to be 1.88 billion in the year 2010. Exchanging views via EMail is pretty simple and convinient. Being able to send files as attachment to an email enables the user to share almost anything which can be comprehended in bits and bytes.\par

Any avid user of email will agree that, while attaching files is sufficient for a one-time transmission it becomes rather unmanagable when the same file is to be transmitted over and again with modifications. For example when students send their project reports to the respective guides or when two individuals collaborate on a report, they generally exchange files with some modifications. Now, in this scenario, after a few such conversations, the directory in which the files were downloaded becomes clumsy with file names like ``report.odt'', ``report2.odt'', ``reportModified.odt'' etc .\par

\subsection{The Social Web}
There has been no looking back since the day Tim Berners Lee came up with the World Wide Web. Although the Web is just an application which runs on the Internet (i.e inter-network or network of networks of computers), to a layman both the terms seem synonymous and are often used interchangibly. In the recent past there has been a significant rise in web applications which let users share different forms of information online.
Social networking portals like Facebook and Google plus enable the common man to speak his mind, communicate with friends living next door and over-seas alike.\par 
Microblogging applications like Twitter and Identica, with millions of users feeding live intel have become way faster than the traditional news networks. Recently, the author felt miniscule tremors in Delhi and shared it on twitter. But upon analysing the feed he found \emph{tweets} of people from Pakistan who were standing outside their houses. Few minutes later the news flashes confirmed that epicentre was somewhere in Pakistan.\par
Video sharing portals like Youtube allow users to share their videos with the whole world. The new era of singers and musicians like Justin Bieber \& Kurt Hugo Schiender and their worldwide fame is the proof of its reach.\par
Although these applications are good at what they do, they do not help users to collaborate.

\subsection{Organised Collaboration}
It is not that there isn't any good solution for collaboration. Many big companies have developed their home-brewed solutions for collaboration to name a few Microsoft Projects, Work Plan etc. Many Enterprise Resource Management software also provide these features along with a plethora of others.\par
As far as source code management is concerned there are a host of applications which make it trivial eg Git, Mercurial etc. The Linux Kernel - the largest distributed software development project is a testimony to the efficiency of Git. As a matter of fact, a vast majority of open source projects have been developed following a similar model by developers contributing from different parts of the world.\par
Now, talking about files other than source code, one can use utilities like Google Docs or ThinkFree's office suite which let users to work on files stored on the cloud, thus ensuring safety against accidental deletion in case of local Operating System failure.

\section{A/B Testing}
DropBox is a web application which enables users to store and share files and folers across the Internet using cloud computing techniques. DropBox achieves its goal by monitoring a specified folder on the drive of a user and updating itself and other users with whom the files have been shared upon any modification. Since there is a critical requirement of continuous, uninterrupted internet connectivity for working on a service like Google Docs is not present in case of DropBox, users always have access to their files. This feature along with its compatibility with all major platforms like Linux, Windows Mac and Android make DropBox the most favoured file sharing and syncing service.

\section{Motivation}
So if there are services which are awesome enough when it comes to file sharing and syncing, what is the motive behind this project? The answer is simple, Need. Although DropBox provides its service both for free and paid, there are situations in which certain situations in which need for an alternative is felt. Lets analyse the reasons:
\begin{enumerate}
  \item There is a limit on the amount of data one can save for free on their server.
  \item Cost of increasing space is substancial.
  \item One needs internet connectivity even if the files are to be shared only on a Local Area Network.
  \item Although cloud computing is one of the pioneering fields of research in the present times, there are organisations which do not trust their data on an external server. eg Amity University does not allow use of Google Docs for official purposes. 
\end{enumerate}
So if an application lets the awesomeness of DropBox to be implemented on a private server, with space as per requirement and internet no bar, wouldn't that be equally awesome?
This was the cause and motivation behind pursuing this project.

\section{Document Structure}
This chapter concentrated on building a base for the problem statement and pitch the statement in favour of the need for a solution. The coming chapters will focus on various aspects of the application.
Chapter \ref{chap:devTools} intends to introduce the various tools used widely for developing the project. Right from the operating system to the Framework, everything is covered.
Chapter \ref{chap:devProc} walks the reader through the development process followed and also highlights the major challenges faced while creating the project and sites the solutions discovered.
The report is well formatted and structured, thanks to the excellent set of utilities provided by \LaTeX. 
